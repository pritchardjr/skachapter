\documentclass[12pt,a4paper]{article}
\usepackage{epsfig}
\usepackage{graphicx}
\usepackage{amssymb}
\usepackage{mathptmx,helvet,courier}
% add your own packages as needed

% Latex Template for AASKA14 Abstract Book
% DO NOT MODIFY THESE MARGINS
\parskip 3.0mm
\textheight 222mm
\textwidth 170mm
\topmargin 10mm
\oddsidemargin -5mm
\pagestyle{empty}

%
% Thanks to the Star Formation Newsletter for the basic template
%

% add your own definitions as needed

\begin{document}

\noindent
{\Large\bf
%% Between these brackets you write the TITLE of your paper:
This is the Title of my Paper.
} % end title

\noindent
%% Here comes the AUTHOR(s) of the paper, please indicate within $^...$
%% the number which correosponds to the institute of each author.
%% The speaker may also be identified explicitly using the * symbol
%% Add authors as needed.
{\em 
Pritchard$^{1,*}$, Ichiki, Mesinger, Metcalf, Pourtsidou, Santos, on behalf of the Cosmology-SWG and EoR/CD-SWG} % end author list

\noindent
%% Here you write your INSTITUTE name(s) and address(es),
%% the number in $^..$ indicates your author number, for example:
$^1$ Imperial College London\\
$^*$ Presenter

\noindent
%% Here you may write the e-mail address of one or more
%% of the authors who will act as contact person, for example:
E-mail contact: j.pritchard@imperial.ac.uk

%% Here you write your ABSTRACT:
SKA Phase 1 will build upon early detections of the EoR by precursor instruments, such as MWA, PAPER, LOFAR, and HERA, to make the first high signal-to-noise measurements of fluctuations in the 21 cm brightness temperature from both reionization and the cosmic dawn. This will allow both imaging and statistical maps of the 21cm signal at redshifts $z=6-30$ and constrain the underlying cosmology and evolution of the density field. This era includes nearly 60\% of the (in principle) observable volume of the Universe and many more linear modes than the CMB, presenting an opportunity for SKA to usher in a new level of precision cosmology. This optimistic picture is complicated by the need to understand and remove the effect of astrophysics, so that systematics rather than statistics will limit constraints.

This chapter will describe the cosmological, as opposed to astrophysical, information available to SKA Phase 1. Key areas for discussion include: cosmological parameters constraints using 21cm fluctuations as a tracer of the density field; lensing of the 21cm signal, constraints on heating via exotic physics such as decaying or annihilating dark matter; impact of fundamental physics such as non-Gaussianity or warm dark matter on the source population; and constraints on the bulk flows arising from the decoupling of baryons and photons at $z=1000$. The chapter will explore the path to separating cosmology from `gastrophysics', for example via velocity space distortions and separation in redshift. We will discuss new opportunities for extracting cosmology made possible by the sensitivity of SKA-1 and explore the advances achievable with SKA-2.

\end{document}
